\documentclass[10pt]{exam}
\firstpageheader{Math 3900 Spring 2019}{Quiz 1 }{Instructor:  J. Haga}
\printanswers
\usepackage{amsfonts}
\usepackage{amsmath}
\usepackage{amssymb, graphicx, enumerate}
\usepackage[framed,indented,numbered,autolinebreaks,useliterate]{mcode}
\begin{document}
$\phantom{x}$

\newcommand{\dd}{\textrm{d}}
\newcommand{\RR}{\mathbb R}

\newcommand{\bfa}{\mathbf a}
\newcommand{\bfb}{\mathbf b}
\newcommand{\bfc}{\mathbf c}
\newcommand{\bfd}{\mathbf d}
\newcommand{\bfe}{\mathbf e}
\newcommand{\bff}{\mathbf f}
\newcommand{\bfg}{\mathbf g}
\newcommand{\bfh}{\mathbf h}
\newcommand{\bfi}{\mathbf i}
\newcommand{\bfj}{\mathbf j}
\newcommand{\bfk}{\mathbf k}
\newcommand{\bfl}{\mathbf l}
\newcommand{\bfm}{\mathbf m}
\newcommand{\bfn}{\mathbf n}
\newcommand{\bfo}{\mathbf o}
\newcommand{\bfp}{\mathbf p}
\newcommand{\bfq}{\mathbf q}
\newcommand{\bfr}{\mathbf r}
\newcommand{\bfs}{\mathbf s}
\newcommand{\bft}{\mathbf t}
\newcommand{\bfu}{\mathbf u}
\newcommand{\bfv}{\mathbf v}
\newcommand{\bfw}{\mathbf w}
\newcommand{\bfx}{\mathbf x}
\newcommand{\bfy}{\mathbf y}
\newcommand{\bfz}{\mathbf z}

\noindent Name: \underline{Charles Arnaudo\hspace{3.5in}}
\vskip 0.5cm

\begin{questions}
\question Use the Taylor series for $f(x)=\frac{1}{x^2}$ around $x_0=1$ to approximate $\frac{1}{1.96}$ to within $10^{-6}$.  To get full credit, you must clearly show how use of the remainder term informs us of the minimal degree of the Taylor expansion required.  What is the interval/radius of convergence for this Taylor series?
\begin{solution}
I expanded the function around 1 because of it's proximity to the value we'd like to approximate. I was able
to bound the remainder term by 1, because for the function and all it's derivatives, the absolute value $\frac{1}{1.96}$
is less than the absolute value at 1. After calculating 7 remainder terms by hand, I decided to use matlab to determine
where the remainder term is less than $10^{-6}$.
\begin{lstlisting}
syms x;
f(x) = 1/x^2;
N = 10^(-6);

for i=1:10^6
    f = diff(f);
    r = abs( (f(1)/factorial(i + 1)) * (1 - 1/1.96)^(i + 1) );
    if r <= N
        i
        vpa(double(r))
        break
    end
end
\end{lstlisting}
The result shows that that $T_{\text{18}}$ will give us an aproximation of $\frac{1}{1.96}$ within $10^{-6}$.
I also used matlab to compute this value.
\begin{lstlisting}
f(x) = 1/x^2;
sum = f(1);
for i=1:19
    f = diff(f);
    sum = sum + ( (1 / factorial(i)) * (1/1.96 - 1)^(i) );
    vpa(double(sum))
end
\end{lstlisting}

$\frac{1}{1.96}$ approximated to $10^{-6}$ is 0.510204. The radius of convergence for this series is x $\neq$ 0.



\end{solution}
\question Determine the Taylor series expansion of $f(x) = x-x^2+4x^3+x^4-x^5$ around $x_0=5$.  What is the interval/radius of convergence for this Taylor series?  Approximate $f(10)$ to within $10^{-8}$.  Is the Taylor series of $f$ an infinite power series?  Draw a conclusion about the Taylor expansions of polynomials and offer an explanation.
\begin{solution}
The Taylor series expansion of $f(x) = x-x^2+4x^3+x^4-x^5$ around $x_0=5$ is
$$ -2020 - 2334 (x - 5) - 1041 (x - 5)^2 - 226 (x - 5)^3 - 24 (x - 5)^4 - (x - 5)^5 $$
The series converges everywhere. The series is an infinite power series. $f(10)$ approximated to within $10^{-8}$
is -84066.0, which isn't right. I used the same code as above to get this value. Each successive
expansion of a polynimial in a Taylor series is a partial sum of the original function, similar to
Riemann sums.
\end{solution}
\end{questions}
\end{document}