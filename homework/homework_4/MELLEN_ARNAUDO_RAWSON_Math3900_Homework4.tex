\documentclass[12 pt, letterpaper]{exam}
\printanswers
\usepackage{amsfonts}
\usepackage{graphicx}
\usepackage{amsthm}
\usepackage{amssymb}
\usepackage{amsmath}
\usepackage{enumerate, mathrsfs}
\usepackage[framed,indented,numbered,autolinebreaks,useliterate]{mcode}


\theoremstyle{definition}
\newtheorem{ex}{Example}
\newtheorem{df}{Definition}
\newtheorem{thm}{Theorem}
\newtheorem{prob}{Problem}

\newcommand{\suchthat}{\,\Big{|}\,}
\newcommand{\ZZ}{\mathbb{Z}}
\newcommand{\QQ}{\mathbb{Q}}
\newcommand{\NN}{\mathbb{N}}
\newcommand{\RR}{\mathbb{R}}
\newcommand{\CC}{\mathbb{C}}
\newcommand{\dd}{\,\,\textrm{d}}

% \firstpageheader{Math 3900 Spring 2018}{Homework 4 }{Instructor:  J. Haga}

\firstpageheader{Math 3900 Spring 2018}{Homework 5}{Arnaudo, Mellen, Rawson}


\begin{document}
\begin{questions}
\question[50] Use equation 3.10 of the text to determine the interpolatory polynomial for the data $$f(1.4) = 2.37\quad \quad f(1.8) = 3.351 \quad \quad f(1.9) = 0.233 \quad \quad f(2.5) = 4.572.$$  Determine the value of this polynomial at $x=2.$  Show all of your steps. 
\begin{solution}
    We started by computing all of the divided differences for our data, starting with the data:

    $f[x_0]=f(x_0)$, $f[x_1]=f(x_1)$, $f[x_2]=f(x_2)$, $f[x_3]=f(x_3)$
    
    and then using the equations:

    \resizebox{\textwidth}{!}{\begin{tabular}{| l | l | l |}
    \hline
     $f[x_0,x_1]=\frac{f[x_1]-f[x_0]}{x_1-x_0}$ & - & - \\ \hline
     $f[x_1,x_2]=\frac{f[x_2]-f[x_1]}{x_2-x_1}$ & $f[x_0,x_1,x_2]=\frac{f[x_1,x_2]-f[x_0,x_1]}{x_2-x_0}$ & - \\ \hline
     $f[x_2,x_3]=\frac{f[x_3]-f[x_2]}{x_3-x_2}$ & $f[x_1,x_2,x_3]=\frac{f[x_2,x_3]-f[x_1,x_2]}{x_3-x_1}$ & $f[x_0,x_1,x_2,x_3]=\frac{f[x_1,x_2,x_3]-f[x_0,x_1,x_2]}{x_3-x_0}$ \\ \hline
    \end{tabular}}

    After computing, we got the divided differences to be:

    \resizebox{\textwidth}{!}{\begin{tabular}{| l | l | l | l | l |}
    \hline
    $x_0=1.4$ & $f[x_0]=2.37$  & - & - & - \\ \hline
    $x_1=1.8$ & $f[x_1]=3.351$ & $f[x_0,x_1]=2.4525$ & - & - \\ \hline
    $x_2=1.9$ & $f[x_2]=0.233$ & $f[x_1,x_2]=-31.18$ & $f[x_0,x_1,x_2]=-67.265$ & - \\ \hline
    $x_3=2.5$ & $f[x_3]=4.572$ & $f[x_2,x_3]=7.23167$ & $f[x_1,x_2,x_3]=54.8738$ & $f[x_0,x_1,x_2,x_3]=111.035$ \\ \hline
    \end{tabular}}

    Given equation 3.10, our interpolatory polynomial is
    $$P_3(x) = 2.37 + 2.4525(x-1.4) - 67.265(x-1.4)(x-1.8) + 111.035(x-1.4)(x-1.8)(x-1.9)$$

    Evaluating $P_3(2)$ gives us $-2.89788$.
\end{solution}

\newpage
\question[50] Follow the pseudo-code expressed in Algorithm 3.2 of the text, to write a MATLAB script which will perform Newton's Divided Difference method to determine the interpolatory polynomial for the data in the previous problem.  Show that your script verifies the value of $P(2)$.

\begin{solution}
    This is our MATLAB code:
    \begin{lstlisting}
xValues = [1.4 1.8 1.9 2.5];
yValues = [2.37 3.351 0.233 4.572];

syms P(x);
P(x) = NIDD(xValues, yValues);
double(P(2))

function P = NIDD(xValues, yValues)
    n = length(xValues);
    F = zeros(n,n);
    F(1:n,1) = yValues;
    for i = 2:n
        for j = 2:i
            F(i,j) = (F(i,j-1)-F(i-1,j-1))/(xValues(i)-xValues((i-j)+1));
        end
    end
 
    syms P(x);
    P = F(1,1);
    for i=2:n
        T = F(i,i);
        for j=1:i-1
            T = T * (x - xValues(j));
        end
        P = P + T;
    end
end
    \end{lstlisting}
    We get
    $$ P(2) = -2.8979,$$
    which is the same as our answer from part 1.
\end{solution}

\end{questions}
\end{document}
