\documentclass[12 pt, letterpaper]{exam}
\usepackage{amsfonts}
\usepackage{graphicx}
\usepackage{amsthm}
\usepackage{amssymb}
\usepackage{amsmath}
\usepackage{enumerate, mathrsfs}
\usepackage[framed,indented,numbered,autolinebreaks,useliterate]{mcode}
\printanswers

\theoremstyle{definition}
\newtheorem{ex}{Example}
\newtheorem{df}{Definition}
\newtheorem{thm}{Theorem}
\newtheorem{prob}{Problem}

\newcommand{\suchthat}{\,\Big{|}\,}
\newcommand{\ZZ}{\mathbb{Z}}
\newcommand{\QQ}{\mathbb{Q}}
\newcommand{\NN}{\mathbb{N}}
\newcommand{\RR}{\mathbb{R}}
\newcommand{\CC}{\mathbb{C}}
\newcommand{\dd}{\,\,\textrm{d}}

%  \firstpageheader{Math 3900 Spring 2019}{Homework 6}{Instructor:  J. Haga}
 
% READ THIS:

% If you're going to adjust this code to formulate your own homework, then please comment (with a %) the above line and uncomment the line below (and change the last names of the mathematicians to your group members' last names in alphabetical order).

\firstpageheader{Math 3900 Spring 2018}{Homework 6}{Arnaudo, Mellen, Rawson}

\begin{document}
\begin{questions}


\question[35] The following MATLAB code generates the free/natural cubic spline coefficients given $x$ and $f$ data:
\begin{lstlisting}
function M=CubicSpline19(x,f)
n=length(x)-1;
h=zeros(n,1);
h=x(2:n+1,1)-x(1:n,1);
C=zeros(n+1);
C(1,1)=1;
C(n+1,n+1)=1;
for j=2:n
    C(j,j-1)=h(j-1);
    C(j,j)=2*(h(j-1)+h(j));
    C(j,j+1)=h(j);
end
v=zeros(n+1,1);
for j=2:n
    v(j)=3/h(j)*(f(j+1)-f(j))-3/h(j-1)*(f(j)-f(j-1));
end
c=C^-1*v;
a=f;
d=zeros(n+1,1);
for i=1:n
    d(i)=(c(i+1)-c(i))/(3*h(i));
end
b=zeros(n+1,1);
for i=1:n
    b(i)=(f(i+1)-f(i))/(h(i))-h(i)*(c(i+1)+2*c(i))/3;
end
M=[a,b,c,d];
\end{lstlisting}
Adjust this code to provide the coefficients for the clamped cubic spline.  You may assume that $f'(x_0)$ and $f'(x_n)$ would be additional data given by the user.
\question[35] The following MATLAB code derives the 5 point midpoint formula:

\begin{lstlisting}
function FivePointMidPoint()

syms c h f0 f1 f2 f3 f4 x
% c denotes the midpoint
% h denotes the spacing between consecutive values of x
% f0, f1, f2, f3, and f4 are the function values at x0, x1, x2, x3, and x4
% (respectively).

% the following vector contains all of the x values
v=c*[1;1;1;1;1]+h*[-2;-1;0;1;2];

% the following vector contains all of the y values
f=[f0;f1;f2;f3;f4];

% we will build the Lagrange coefficient polynomials
LCP=[];
for n=0:4
    L=1;
    for j=0:4
        if j~=n
            L=L*(x-v(j+1))/(v(n+1)-v(j+1));
        end
    end
    LCP=[LCP;L];
end

% we now express the Lagrange Interpolating Polynomial as an appropriate linear combination of the coefficient polynomials
L=sum(LCP.*f);

% we differentiate the Lagrange Interpolating Polynomial
dL=diff(L,x);

% the formula comes from evaluating this derivative at c
formula=subs(dL,c)
\end{lstlisting}
Adjust this code to provide seven point midpoint and endpoint formulae.
\begin{solution}

The following code will provide the seven point mid point formula.
\begin{lstlisting}
function SevenPointMidPoint()

syms c h f0 f1 f2 f3 f4 f5 f6 x
% c denotes the midpoint
% h denotes the spacing between consecutive values of x
% f0, f1, f2, f3, f4, f5 and f6 are the function values at x0, x1, x2, x3 x4, x5 and x6
% (respectively).

% the following vector contains all of the x values
v=c*[1;1;1;1;1;1;1]+h*[-3;-2;-1;0;1;2;3];

% the following vector contains all of the y values
f=[f0;f1;f2;f3;f4;f5;f6];

% we will build the Lagrange coefficient polynomials
LCP=[];
for n=0:6
    L=1;
    for j=0:6
        if j~=n
            L=L*(x-v(j+1))/(v(n+1)-v(j+1));
        end
    end
    LCP=[LCP;L];
end

% we now express the Lagrange Interpolating Polynomial as an appropriate linear combination of the coefficient polynomials
L=sum(LCP.*f);
% we differentiate the Lagrange Interpolating Polynomial
dL=diff(L,x);
% the formula comes from evaluating this derivative at c
formula=subs(dL,c)
\end{lstlisting}
This gives the formula: 
$$ \frac{(3_1)}{(20h)} - \frac{f_0}{(60h)} - \frac{(3f_2)}{(4h)} + \frac{(3f_4)}{(4h)} - \frac{(3f_5)}{(20h)} + \frac{f_6}{(60h)} $$

The following code will provide the seven point end point formula.
\begin{lstlisting}
function SevenPointMidPoint()

syms c h f0 f1 f2 f3 f4 f5 f6 x
% c denotes the endpoint
% h denotes the spacing between consecutive values of x
% f0, f1, f2, f3, f4, f5 and f6 are the function values at x0, x1, x2, x3 x4, x5 and x6
% (respectively).

% the following vector contains all of the x values
v=c*[1;1;1;1;1;1;1]+h*[0;1;2;3;4;5;6];

% the following vector contains all of the y values
f=[f0;f1;f2;f3;f4;f5;f6];

% we will build the Lagrange coefficient polynomials
LCP=[];
for n=0:6
    L=1;
    for j=0:6
        if j ~= n
            L=L*(x-v(j+1))/(v(n+1)-v(j+1));
        end
    end
    LCP=[LCP;L];
end

% we now express the Lagrange Interpolating Polynomial as an appropriate linear combination of the coefficient polynomials
L=sum(LCP.*f);
% we differentiate the Lagrange Interpolating Polynomial
dL=diff(L,x);
% the formula comes from evaluating this derivative at c
formula=subs(dL,c)    
\end{lstlisting}
This gives the formula:
$$ \frac{(6f_1)}{h} - \frac{(49f_0)}{(20h)} - \frac{(15f_2)}{(2h)} + \frac{(20f_3)}{(3h)} - \frac{(15f_4)}{(4h)} + \frac{(6f_5)}{(5h)} - \frac{f_6}{(6h)} $$
\end{solution}

\question[30] Use the 7 point midpoint formula derived in the previous question to estimate $f'(2.4)$, where $f(x) = x\sin(x)$, and the function is sampled at $$x=2.1,\,2.2,\,2.3, \,2.4, \,2.5, \,2.6, \,2.7.$$  Determine the absolute error in this approximation.  You may use MATLAB here.
\begin{solution}
We used the following code to calcuate the error.
\begin{lstlisting}
syms f(x);
f(x) = x * sin(x);
fp(x) = diff(f, x);

spm = (3*f(2.2))/(20*.1) - f(2.1)/(60*.1) - (3*f(2.3))/(4*.1) + (3*f(2.5))/(4*.1) - (3*f(2.6))/(20*.1) + f(2.7)/(60*.1);
vpa(spm)
vpa(fp(2.4))

err = abs(vpa(spm) - vpa(fp(2.4)))    
\end{lstlisting}
This shows the absolute error in using the seven point midpoint formula to approximate $2.4 * sin(2.4)$ is 
$$ 0.000000021072345782420434778549084395847 $$
\end{solution}
\end{questions}
\end{document}
